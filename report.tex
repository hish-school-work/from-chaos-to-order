\documentclass[12pt]{article}
\usepackage[margin=1in]{geometry} % Sets all margins to 1 inch
\usepackage{amsmath}
\usepackage{graphicx}
\usepackage{hyperref}

\title{From Chaos to Order\\
       \large Modeling Natural Phenomena with Fractals}
\author{Hishmat (Hish) Salehi - 1812352}
\date{\today}

\begin{document}

\maketitle

\begin{abstract}
This project, titled \textit{From Chaos to Order}, aims to explore the integration of Chaos Theory and Fractal Geometry in modeling complex natural phenomena. By leveraging the principles of chaos and the self-similar structures of fractals, the project seeks to develop mathematical models that can accurately represent and predict patterns observed in nature, such as plant growth, river networks, and weather systems. The study will involve developing and analyzing these models through computational simulations, ultimately bridging the gap between theoretical mathematics and real-world applications.
\end{abstract}

\tableofcontents

\section{Background Knowledge}

\subsection{What is Chaos Theory?}
Chaos Theory is a branch of mathematics focusing on the behavior of dynamical systems that are highly sensitive to initial conditions. This phenomenon is popularly referred to as the butterfly effect.

\subsection{What is the Butterfly Effect?}
TODO

\subsection{What are Fractals?}
TODO

\end{document}
