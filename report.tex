\documentclass[12pt]{article}
\usepackage[margin=1in]{geometry}
\usepackage{setspace}
\doublespacing
\usepackage{amsmath}
\usepackage{graphicx}
\usepackage{hyperref}

\title{From Chaos to Order\\
       \large The Fractal Geometry of Our World}
\author{Hishmat (Hish) Salehi \\
\href{mailto:hishmat@ualberta.ca}{hishmat@ualberta.ca}: 1812352}
\date{\today}

\begin{document}

\maketitle

\begin{center}
\textbf{Math/STAT 900: Capstone Project Report}  \\
\emph{Supervised by Professor Michael Y Li - 
\href{mailto:myli@ualberta.ca}{myli@ualberta.ca}}  \\
University of Alberta \\
Mathematics \& Statistical Sciences
\end{center}

\begin{abstract}
This paper is the result of my burning curiosity about Chaos Theory and Fractals and acts as an introduction to both, in hopes that it sparks some curiosity for you as well. I aim to explore the fascinating connection between Chaos Theory and Fractal Geometry in nature. 

I will begin by providing the background you will need to understand the mathematics involved to follow along. Then, I will demonstrate the incredible power of mathematics to understand the complexities of the natural world, illustrated by the complex patterns we see in coastlines, trees, and mountains. I will conclude the paper with some philosophical insights—that there is no order without chaos—and two perspectives to consider when reflecting on the unpredictable beauty of nature.
\end{abstract}

\newpage

\tableofcontents

\section{Background}
To appreciate Chaos Theory and Fractals, it is essential to understand some fundamental concepts in mathematics and systems theory.

\subsection{Functions and Systems}
A function is a mathematical rule that assigns each input exactly one output. Functions are foundational in describing relationships between variables. For example, \( f(x) = x + 2 \) maps the input \( x = 3 \) to the output \( f(3) = 5 \).

A system, in this context, refers to a collection of functions and equations that interact with each other. Systems can range from simple pairs of equations to highly complex networks like weather patterns, where variables such as temperature, humidity, and wind interact dynamically.

\subsection{Deterministic and Complex Systems}
A deterministic system operates under precise rules, meaning that given an initial state, its future behavior can be predicted exactly. Classical mechanics, which governs the motion of planets and pendulums, is an example of deterministic systems.

In contrast, complex systems consist of numerous interconnected components whose interactions lead to emergent behavior that is difficult to predict. These systems exhibit properties such as nonlinearity and feedback loops, making their long-term behavior hard to forecast.

\section{Introduction}

\subsection{Chaos Theory}
Chaos theory is the study of complex systems that appear random, but in reality are governed by underlying deterministic laws. This means that even though their behavior seems unpredictable, it's driven by precise rules. The key feature of chaos is "sensitivity to initial conditions," meaning tiny changes at the start can lead to vastly different outcomes. This phenomenon is also known as the "butterfly effect."

\subsection{The Butterfly Effect}
The butterfly effect is the idea that small actions, like the flap of a butterfly’s wings, can cause large, unexpected changes, like altering weather patterns far away. It shows how tiny differences in starting conditions can lead to vastly different outcomes even in a deterministic system, where a very small difference in the beginning can make predictions difficult. Mathematically, this is expressed through exponential divergence of trajectories in phase space.

\subsection{Chaos vs. Classical Mechanics}
Classical mechanics is the study of systems that behave in a predictable way, like the motion of planets or a swinging pendulum. In classical mechanics, if you know the starting conditions (like the position and speed), you can predict exactly what will happen next. Chaos Theory, on the other hand, studies systems that can’t be predicted easily, even if they follow precise rules. Small changes at the start can lead to wildly different outcomes, making long-term predictions nearly impossible.

This sensitivity to initial conditions is what sets chaotic systems apart from the predictable systems studied in classical mechanics. Chaos Theory helps us understand that just because a system has rules doesn’t mean it’s easy to predict, especially when those rules cause big changes from tiny variations.


\subsection{Attractors in Dynamical Systems}
An attractor is a pattern or path that a system’s motion tends to follow over time. In classical mechanics, an attractor can be a single point (like an object that has come to a complete stop), a closed loop (a repeating cycle), or a torus (a combination of cycles). Attractors help scientists visualize the behavior of systems and see if they tend toward stable and predictable paths.

\subsection{Strange Attractors and Chaos}
Strange attractors are a special type of attractor found in chaotic systems. Unlike traditional attractors, strange attractors create paths that are detailed and never repeat, even though they follow specific rules. The discovery of strange attractors helped scientists understand how chaotic systems can look random but still follow hidden structures. This complexity led to more questions about how these patterns could be studied and visualized.

Strange attractors reveal that even in chaotic systems, there is an underlying structure that governs their behavior. These attractors create intricate patterns that appear random at first but show order upon closer examination. A key feature of strange attractors is their self-similarity, meaning their patterns repeat at different scales. This "self-similarity" shows that chaotic systems have repeating structures, even in their randomness. These detailed patterns in strange attractors gave rise to the concept of fractals.

\section{Fractal Geometry}
\subsection{Understanding Fractals}
Fractals are complex shapes with self-similarity, meaning they look similar at different scales of magnification. In Chaos Theory, fractals help describe the repeating patterns and hidden structures found in chaotic systems, like strange attractors. The idea of fractals transformed how scientists visualize and understand complexity, allowing them to model natural systems, from coastlines and mountains to clouds and even plant growth.

\subsection{What is self-similarity in fractals?}
Self-similarity in fractals means that the same patterns repeat at different scales. No matter how much you zoom in or out, you will see similar shapes and structures. For example, the branches of a tree look like smaller versions of the entire tree, and this is a form of self-similarity. In fractals, this property shows how simple rules can create complex and repeating structures.

\subsection{How do fractals exhibit infinite detail?}
Fractals exhibit infinite detail because their patterns continue to repeat endlessly, no matter how much you zoom in. Each part of the fractal reveals smaller and smaller structures that look similar to the whole. This means you can keep exploring deeper levels of a fractal, and there will always be more intricate details to discover. This infinite complexity makes fractals useful for modeling natural systems like coastlines, mountains, and clouds.

\subsection{What is fractal dimension?}
Fractal dimension is a way of measuring how complex a fractal is. Unlike regular shapes like lines (1D), squares (2D), or cubes (3D), fractals can have dimensions that are not whole numbers. For example, a fractal might have a dimension of 1.5, meaning it is more than a line but less than a flat surface. This helps describe how much space a fractal's pattern fills as you zoom in. Fractal dimension shows the unique, in-between nature of fractals and their ability to capture complexity.

\section{Mathematics of Fractals}

\subsection{How to calculate the fractal dimension of a shape?}
Fractal dimension measures how a fractal scales differently from traditional shapes like lines, squares, or cubes. The key idea is to understand how the number of parts (\(N\)) changes as the size of each part (\(\varepsilon\)) is scaled. The fractal dimension (\(D\)) is calculated using the formula:
\[
D = \frac{\log(N)}{\log(1/\varepsilon)}.
\]

\subsubsection{Example}
\begin{enumerate}
    \item For a simple line:
	\begin{enumerate}
		\item Divide the line into smaller segments of size \(1/3\).
		\item You’ll find \(N = 3\), so \(D = 1\) (a one-dimensional shape).
\end{enumerate}
    \item For a fractal, like the Koch snowflake:
	\begin{enumerate}
		\item Divide into segments of size \(1/3\).
		\item You’ll find \(N = 4\), so \(D = \frac{\log(4)}{\log(3)} \approx 1.2619\).
		\item This non-integer dimension reflects the fractal's complexity, lying between one and two dimensions.
	\end{enumerate}
\end{enumerate}


\subsection{How do fractals work mathematically?}
Fractals are created using iterative processes, which means repeating a simple rule over and over. Mathematically, this is often done using equations or algorithms that generate self-similar patterns. Here’s how it works:
\begin{itemize}
    \item Start with a basic shape or rule, called the "seed." For example, a straight line can be the starting point.
    \item Apply a transformation to the seed, such as bending, splitting, or scaling it.
    \item Repeat the transformation multiple times. Each step creates a new, more detailed version of the fractal.
\end{itemize}

For example, the famous Koch snowflake starts with an equilateral triangle. In each iteration, the middle third of each line segment is replaced with two segments forming a smaller triangle. Repeating this process indefinitely creates a fractal with infinite detail and a finite area.

Fractals also rely on recursive mathematical functions, such as:
\[
z_{n+1} = z_n^2 + c
\]
This equation generates fractals like the Mandelbrot set by iterating \( z \) and plotting the result. The repeated application of rules creates the complexity and self-similarity seen in fractals, showing how simple math can lead to infinite beauty.

\subsection{How are iterated maps used to generate fractals?}
Iterated maps are used to generate fractals by repeatedly applying a mathematical rule or transformation to a starting value. Each iteration produces a new result, and these results together form a fractal pattern. The process works like this:
\begin{itemize}
    \item Start with an initial value, often called the "seed."
    \item Apply a mapping function, such as \( f(x) = x^2 + c \), to the seed.
    \item Take the result and use it as the input for the next iteration.
    \item Continue this process for many iterations, plotting the results.
\end{itemize}

For example, the Mandelbrot set is created using the iterated map:
\[
z_{n+1} = z_n^2 + c
\]
where \( z \) is a complex number and \( c \) is a constant. By iterating this map and checking whether the results stay bounded or diverge, we can generate the intricate and self-similar shapes that characterize fractals.

\subsection{What were Mandelbrot's contributions to fractal geometry?}
Benoît Mandelbrot is considered the father of fractal geometry. His contributions include:
\begin{itemize}
    \item Introducing the term "fractal" to describe shapes that have self-similarity and infinite detail.
    \item Developing the Mandelbrot set, a famous fractal generated by iterating the equation \( z_{n+1} = z_n^2 + c \). This set visualized how simple mathematical rules can create stunningly complex and beautiful patterns.
    \item Applying fractals to real-world phenomena, such as coastlines, mountains, and financial markets, showing that fractals can model irregular shapes and behaviors found in nature.
    \item Using computer graphics to explore and popularize fractals, making their beauty and complexity accessible to scientists, mathematicians, and the general public.
\end{itemize}
Mandelbrot's work transformed our understanding of geometry by showing that simple rules can describe the complexity of the natural world.

\subsection{How does bifurcation theory explain the transition from regular to chaotic behavior?}
Bifurcation theory studies how small changes in a system's parameters can cause sudden shifts in its behavior. In a dynamical system, a bifurcation occurs when a steady or predictable pattern changes into something new, often leading to chaos. This transition works like this:
\begin{itemize}
    \item Start with a system that behaves regularly, such as a pendulum swinging back and forth in a predictable way.
    \item Gradually change a key parameter, like the force acting on the pendulum.
    \item At certain critical values, the system's behavior changes abruptly. For example, it might start oscillating in a complex or unpredictable way.
    \item As the parameter continues to change, the system may become fully chaotic, showing sensitive dependence on initial conditions.
\end{itemize}

One famous example is the logistic map:
\[
x_{n+1} = r x_n (1 - x_n)
\]
As the parameter \( r \) increases, the system transitions from stable fixed points to periodic cycles and eventually to chaotic behavior. Bifurcation theory helps explain these transitions, showing how simple systems can develop complex and chaotic dynamics.


\section{Fractals in Nature}
\subsection{How do chaotic processes contribute to fractal patterns in nature?}
TODO

\subsection{How do coastlines display fractal patterns?}
TODO

\subsection{How do trees display fractal patterns?}
TODO

\subsection{How do mountains display fractal patterns?}
TODO

\section{Conclusion}

The dynamics of a system at each moment of time can be in one of these two states:
\begin{itemize}
    \item Chaos (unstable)
    \item Order (stable)
\end{itemize}


At either of those states you also need a perspective to be able to maximize your effectiveness and live optimally. These perspectives are Zooming out and Zooming in. 

You zoom out when the system is in a state of chaos. What that means is you try to grasp the bigger picture and understand why things unfold in the long run.

You zoom in when the system is in a state of order. What that means is you bring yourself to the present moment and try to take it in as much as possible. This includes when the economy is stable, when routines are predictable, and when life feels steady.

Chaos theory teaches us that there is no certainty in life, only possibility and patterns, and that is enough.


\nocite{*} 
\bibliographystyle{plain} 
\bibliography{references}

\end{document}
