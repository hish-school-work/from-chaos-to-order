\documentclass[12pt]{article}
\usepackage[margin=1in]{geometry} % Sets all margins to 1 inch
\usepackage{setspace}
\doublespacing
\usepackage{amsmath}
\usepackage{graphicx}
\usepackage{hyperref}

\title{From Chaos to Order\\
       \large The Fractal Geometry of Our World}
\author{Hishmat (Hish) Salehi \\
\href{mailto:hishmat@ualberta.ca}{hishmat@ualberta.ca}: 1812352}
\date{\today}

\begin{document}

\maketitle

\begin{center}
\textbf{Math/STAT 900: Capstone Project Report}  \\
\emph{Supervised by Professor Michael Y Li - 
\href{mailto:myli@ualberta.ca}{myli@ualberta.ca}}  \\
University of Alberta \\
Mathematics \& Statistical Sciences
\end{center}

\begin{abstract}
This paper is the result of my burning curiosity about Chaos Theory and Fractals and acts as an introduction to both, in hopes that it sparks some curiosity for you as well. I aim to explore the fascinating connection between Chaos Theory and Fractal Geometry in nature. 

I will begin by providing the background you will need to understand the mathematics involved to follow along. Then, I will demonstrate the incredible power of mathematics to understand the complexities of the natural world, illustrated by the complex patterns we see in coastlines, trees, and mountains. I will conclude the paper with some philosophical insights—that there is no order without chaos—and two perspectives to consider when reflecting on the unpredictable beauty of nature.
\end{abstract}

\newpage

\tableofcontents

\section{Background}

\subsection{What is a function?}
A function is a rule that connects one number to another in a specific way. Think of it like a machine: you put one number in, the function processes it, and then gives you a result. In mathematical terms, if we have a function \( f \) and we put in a value \( x \), we get an output \( f(x) \). For example, if \( f(x) = x + 2 \), putting in \( x = 3 \) would give an output of \( f(3) = 5 \). Functions are essential because they allow us to see how one thing affects another.

\subsection{What is an equation?}
An equation is a statement that shows two things are equal. It’s written with an equal sign ( = ) between them. For instance, \( x + 2 = 5 \) is an equation that says “\( x + 2 \)” is the same as “5.” Equations are used to express relationships between numbers, variables, or functions, helping us solve for unknowns and describe how quantities are related.

\subsection{What is a system?}
A system is a collection of related parts that interact with each other. In mathematics and science, a system can be as simple as a pair of connected equations or as complex as an entire weather pattern. Each part of the system affects the others, and together they create a larger picture. For example, in a weather system, temperature, humidity, and wind all interact to shape the weather.

\subsection{What is a deterministic system?}
A deterministic system is one where everything happens according to specific rules, so if you know the starting conditions, you can predict what will happen next. For example, if you throw a ball with a certain force and angle, physics laws allow you to calculate exactly where it will land. This kind of system follows rules so precisely that knowing the initial conditions tells you the outcome.

\section{Introduction}

\subsection{What is Chaos Theory?}

Chaos theory is the study of complex systems that appear random, but in reality are governed by underlying deterministic laws. This means that even though their behavior seems unpredictable, it's driven by precise rules. The key feature of chaos is "sensitivity to initial conditions," meaning tiny changes at the start can lead to vastly different outcomes. This phenomenon is also known as the "butterfly effect."

\subsection{What is the Butterfly Effect?}

The butterfly effect is the idea that small actions, like the flap of a butterfly’s wings, can cause large, unexpected changes, like altering weather patterns far away. It shows how tiny differences in starting conditions can lead to vastly different outcomes even in a deterministic system, where a very small difference in the beginning can make predictions difficult.

\subsection{What is the role of strange attractors in chaotic systems?}
TODO

\subsection{Why is nonlinearity important for chaotic behavior?}
TODO

\subsection{How can complex patterns emerge from simple systems?}
TODO

\subsection{What is the relationship between chaos and fractals?}
TODO

\section{Fractals}
\subsection{What are fractals?}
TODO

\subsection{What is self-similarity in fractals?}
TODO

\subsection{How do fractals exhibit infinite detail?}
TODO

\subsection{What is fractal dimension?}
TODO

\subsection{What are the mathematical tools used in fractal geometry?}
TODO

\subsection{How are iterated maps used to generate fractals?}
TODO

\subsection{What were Mandelbrot's contributions to fractal geometry?}
TODO

\subsection{How does bifurcation theory explain the transition from regular to chaotic behaviour?}
TODO

\section{Fractals in Nature}
\subsection{How do chaotic processes contribute to fractal patterns in nature?}
TODO

\subsection{How do coastlines display fractal patterns?}
TODO

\subsection{How do trees display fractal patterns?}
TODO

\subsection{How do mountains display fractal patterns?}
TODO

\section{Conclusion}

The dynamics of a system at each moment of time can be in one of these two states:
\begin{itemize}
    \item Chaos (unstable)
    \item Order (stable)
\end{itemize}


At either of those states you also need a perspective to be able to maximize your effectiveness and live optimally. These perspectives are Zooming out and Zooming in. 

You zoom out when the system is in a state of chaos. What that means is you try to grasp the bigger picture and understand why things unfold in the long run.

You zoom in when the system is in a state of order. What that means is you bring yourself to the present moment and try to take it in as much as possible. This includes when the economy is stable, when routines are predictable, and when life feels steady.

Chaos theory teaches us that there is no certainty in life, only possibility and patterns, and that is enough.


\nocite{*} 
\bibliographystyle{plain} 
\bibliography{references}

\end{document}
