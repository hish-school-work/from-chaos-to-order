\documentclass[12pt]{article}
\usepackage[margin=1in]{geometry} % Sets all margins to 1 inch
\usepackage{setspace}
\doublespacing
\usepackage{amsmath}
\usepackage{graphicx}
\usepackage{hyperref}

\title{From Chaos to Order\\
       \large The Fractal Geometry of Our World}
\author{Hishmat (Hish) Salehi \\
\href{mailto:hishmat@ualberta.ca}{hishmat@ualberta.ca}: 1812352}
\date{\today}

\begin{document}

\maketitle

\begin{center}
\textbf{Math/STAT 900: Capstone Project Report}  \\
\emph{Supervised by Professor Michael Y Li - 
\href{mailto:myli@ualberta.ca}{myli@ualberta.ca}}  \\
University of Alberta \\
Mathematics \& Statistical Sciences
\end{center}

\begin{abstract}
This paper is the result of my burning curiosity about Chaos Theory and Fractals and acts as an introduction to both, in hopes that it sparks some curiosity for you as well. I aim to explore the fascinating connection between Chaos Theory and Fractal Geometry in nature. 

I will begin by providing the background you will need to understand the mathematics involved to follow along. Then, I will demonstrate the incredible power of mathematics to understand the complexities of the natural world, illustrated by the complex patterns we see in coastlines, trees, and mountains. I will conclude the paper with some philosophical insights—that there is no order without chaos—and two perspectives to consider when reflecting on the unpredictable beauty of nature.
\end{abstract}

\newpage

\tableofcontents

\section{Background}

TODO 

\section{Introduction}

\subsection{What is Chaos Theory?}

Chaos theory is the study of complex systems that appear random, but in reality are governed by underlying deterministic laws. This means that even though their behavior seems unpredictable, it's driven by precise rules. The key feature of chaos is "sensitivity to initial conditions," meaning tiny changes at the start can lead to vastly different outcomes. This phenomenon is also known as the "butterfly effect."

\subsection{What is the Butterfly Effect?}

The butterfly effect is the idea that small actions, like the flap of a butterfly’s wings, can cause large, unexpected changes, like altering weather patterns far away. It shows how tiny differences in starting conditions can lead to vastly different outcomes.

\subsection{What is the role of strange attractors in chaotic systems?}
TODO

\subsection{Why is nonlinearity important for chaotic behavior?}
TODO

\subsection{How can complex patterns emerge from simple systems?}
TODO

\subsection{What is the relationship between chaos and fractals?}
TODO

\section{Fractals}
\subsection{What are fractals?}
TODO

\subsection{What is self-similarity in fractals?}
TODO

\subsection{How do fractals exhibit infinite detail?}
TODO

\subsection{What is fractal dimension?}
TODO

\subsection{What are the mathematical tools used in fractal geometry?}
TODO

\subsection{How are iterated maps used to generate fractals?}
TODO

\subsection{What were Mandelbrot's contributions to fractal geometry?}
TODO

\subsection{How does bifurcation theory explain the transition from regular to chaotic behaviour?}
TODO

\section{Fractals in Nature}
\subsection{How do chaotic processes contribute to fractal patterns in nature?}
TODO

\subsection{How do coastlines display fractal patterns?}
TODO

\subsection{How do trees display fractal patterns?}
TODO

\subsection{How do mountains display fractal patterns?}
TODO

\section{Conclusion}

The dynamics of a system at each moment of time can be in one of these two states:
\begin{itemize}
    \item Chaos (unstable)
    \item Order (stable)
\end{itemize}


At either of those states you also need a perspective to be able to maximize your effectiveness and live optimally. These perspectives are Zooming out and Zooming in. 

You zoom out when the system is in a state of chaos. What that means is you try to grasp the bigger picture and understand why things unfold in the long run.

You zoom in when the system is in a state of order. What that means is you bring yourself to the present moment and try to take it in as much as possible. This includes when the economy is stable, when routines are predictable, and when life feels steady.

Chaos theory teaches us that there is no certainty in life, only possibility and patterns, and that is enough.


\nocite{*} 
\bibliographystyle{plain} 
\bibliography{references}

\end{document}
