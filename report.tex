\documentclass[12pt]{article}
\usepackage[margin=1in]{geometry} % Sets all margins to 1 inch
\usepackage{setspace}
\doublespacing
\usepackage{amsmath}
\usepackage{graphicx}
\usepackage{hyperref}

\title{From Chaos to Order\\
       \large The Fractal Geometry of Our World}
\author{Hishmat (Hish) Salehi \\
\href{mailto:hishmat@ualberta.ca}{hishmat@ualberta.ca}: 1812352}
\date{\today}

\begin{document}

\maketitle

\begin{center}
\textbf{Math/STAT 900: Capstone Project Report}  \\
\emph{Supervised by Professor Michael Y Li - 
\href{mailto:myli@ualberta.ca}{myli@ualberta.ca}}  \\
University of Alberta \\
Mathematics \& Statistical Sciences
\end{center}

\begin{abstract}
This paper is the result of my burning curiosity about Chaos Theory and Fractals and acts as an introduction to both, in hopes that it sparks some curiosity for you as well. I aim to explore the fascinating connection between Chaos Theory and Fractal Geometry in nature. 

I will begin by providing the background you will need to understand the mathematics involved to follow along. Then, I will demonstrate the incredible power of mathematics to understand the complexities of the natural world, illustrated by the complex patterns we see in coastlines, trees, and mountains. I will conclude the paper with some philosophical insights—that there is no order without chaos—and two perspectives to consider when reflecting on the unpredictable beauty of nature.
\end{abstract}

\newpage

\tableofcontents

\section{Background}

\subsection{What is a function?}
A function is a rule that connects one number to another in a specific way. Think of it like a machine: you put one number in, the function processes it, and then gives you a result. In mathematical terms, if we have a function \( f \) and we put in a value \( x \), we get an output \( f(x) \). For example, if \( f(x) = x + 2 \), putting in \( x = 3 \) would give an output of \( f(3) = 5 \). Functions are essential because they allow us to see how one thing affects another.

\subsection{What is an equation?}
An equation is a statement that shows two things are equal. It’s written with an equal sign ( = ) between them. For instance, \( x + 2 = 5 \) is an equation that says “\( x + 2 \)” is the same as “5.” Equations are used to express relationships between numbers, variables, or functions, helping us solve for unknowns and describe how quantities are related.

\subsection{What is a system?}
A system is a collection of related parts that interact with each other. In mathematics and science, a system can be as simple as a pair of connected equations or as complex as an entire weather pattern. Each part of the system affects the others, and together they create a larger picture. For example, in a weather system, temperature, humidity, and wind all interact to shape the weather.

\subsection{What is a deterministic system?}
A deterministic system is one where everything happens according to specific rules, so if you know the starting conditions, you can predict what will happen next. For example, if you throw a ball with a certain force and angle, physics laws allow you to calculate exactly where it will land. This kind of system follows rules so precisely that knowing the initial conditions tells you the outcome.

\subsection{What is a complex system?}
A complex system is a group of interconnected parts that interact with each other in ways that are difficult to predict. These systems are made up of many components that influence each other, often leading to unexpected behaviors. Examples of complex systems include ecosystems, weather patterns, and even traffic flows. What makes them complex is that the whole system behaves in ways that cannot be understood by looking at just one part; instead, the interactions between the parts create the overall behavior.

\section{Introduction}

\subsection{What is Chaos Theory?}
Chaos theory is the study of complex systems that appear random, but in reality are governed by underlying deterministic laws. This means that even though their behavior seems unpredictable, it's driven by precise rules. The key feature of chaos is "sensitivity to initial conditions," meaning tiny changes at the start can lead to vastly different outcomes. This phenomenon is also known as the "butterfly effect."

\subsection{What is the Butterfly Effect?}
The butterfly effect is the idea that small actions, like the flap of a butterfly’s wings, can cause large, unexpected changes, like altering weather patterns far away. It shows how tiny differences in starting conditions can lead to vastly different outcomes even in a deterministic system, where a very small difference in the beginning can make predictions difficult.

\subsection{How does Chaos Theory differ from classical mechanics and predictable systems?}
Classical mechanics is the study of systems that behave in a predictable way, like the motion of planets or a swinging pendulum. In classical mechanics, if you know the starting conditions (like the position and speed), you can predict exactly what will happen next. Chaos Theory, on the other hand, studies systems that can’t be predicted easily, even if they follow precise rules. Small changes at the start can lead to wildly different outcomes, making long-term predictions nearly impossible.

This sensitivity to initial conditions is what sets chaotic systems apart from the predictable systems studied in classical mechanics. Chaos Theory helps us understand that just because a system has rules doesn’t mean it’s easy to predict, especially when those rules cause big changes from tiny variations.

\subsection{What is an attractor, and how does it describe motion in a system?}
An attractor is a pattern or path that a system’s motion tends to follow over time. In classical mechanics, an attractor can be a single point (like an object that has come to a complete stop), a closed loop (a repeating cycle), or a torus (a combination of cycles). Attractors help scientists visualize the behavior of systems and see if they tend toward stable and predictable paths.

\subsection{What are strange attractors, and how do they relate to chaotic systems?}
Strange attractors are a special type of attractor found in chaotic systems. Unlike traditional attractors, strange attractors create paths that are detailed and never repeat, even though they follow specific rules. The discovery of strange attractors helped scientists understand how chaotic systems can look random but still follow hidden structures. This complexity led to more questions about how these patterns could be studied and visualized.

\subsection{How do strange attractors reveal hidden structure in chaotic systems?}
Strange attractors reveal that even in chaotic systems, there is an underlying structure that governs their behavior. These attractors create intricate patterns that appear random at first but show order upon closer examination. A key feature of strange attractors is their self-similarity, meaning their patterns repeat at different scales. This "self-similarity" shows that chaotic systems have repeating structures, even in their randomness. These detailed patterns in strange attractors gave rise to the concept of fractals.

\section{Fractals}
\subsection{What are fractals?}
Fractals are complex shapes with self-similarity, meaning they look similar at different scales of magnification. In Chaos Theory, fractals help describe the repeating patterns and hidden structures found in chaotic systems, like strange attractors. The idea of fractals transformed how scientists visualize and understand complexity, allowing them to model natural systems, from coastlines and mountains to clouds and even plant growth.

\subsection{What is self-similarity in fractals?}
Self-similarity in fractals means that the same patterns repeat at different scales. No matter how much you zoom in or out, you will see similar shapes and structures. For example, the branches of a tree look like smaller versions of the entire tree, and this is a form of self-similarity. In fractals, this property shows how simple rules can create complex and repeating structures.

\subsection{How do fractals exhibit infinite detail?}
Fractals exhibit infinite detail because their patterns continue to repeat endlessly, no matter how much you zoom in. Each part of the fractal reveals smaller and smaller structures that look similar to the whole. This means you can keep exploring deeper levels of a fractal, and there will always be more intricate details to discover. This infinite complexity makes fractals useful for modeling natural systems like coastlines, mountains, and clouds.

\subsection{What is fractal dimension?}
Fractal dimension is a way of measuring how complex a fractal is. Unlike regular shapes like lines (1D), squares (2D), or cubes (3D), fractals can have dimensions that are not whole numbers. For example, a fractal might have a dimension of 1.5, meaning it is more than a line but less than a flat surface. This helps describe how much space a fractal's pattern fills as you zoom in. Fractal dimension shows the unique, in-between nature of fractals and their ability to capture complexity.

\subsection{How can we calculate the fractal dimension of a shape?}
To calculate the fractal dimension of a shape, we use methods that measure how the pattern changes as we zoom in or increase the scale. One common method is the box-counting method:
\begin{itemize}
    \item Cover the fractal with a grid of equally sized boxes.
    \item Count how many boxes contain part of the fractal.
    \item Repeat this process with smaller and smaller boxes.
    \item Use the formula \( D = \frac{\log(N)}{\log(1/r)} \), where \( N \) is the number of boxes that contain part of the fractal, and \( r \) is the size of each box relative to the original grid.
\end{itemize}

\subsubsection{Example}
TODO

\subsection{How do fractals work mathematically?}
Fractals are created using iterative processes, which means repeating a simple rule over and over. Mathematically, this is often done using equations or algorithms that generate self-similar patterns. Here’s how it works:
\begin{itemize}
    \item Start with a basic shape or rule, called the "seed." For example, a straight line can be the starting point.
    \item Apply a transformation to the seed, such as bending, splitting, or scaling it.
    \item Repeat the transformation multiple times. Each step creates a new, more detailed version of the fractal.
\end{itemize}

For example, the famous Koch snowflake starts with an equilateral triangle. In each iteration, the middle third of each line segment is replaced with two segments forming a smaller triangle. Repeating this process indefinitely creates a fractal with infinite detail and a finite area.

Fractals also rely on recursive mathematical functions, such as:
\[
z_{n+1} = z_n^2 + c
\]
This equation generates fractals like the Mandelbrot set by iterating \( z \) and plotting the result. The repeated application of rules creates the complexity and self-similarity seen in fractals, showing how simple math can lead to infinite beauty.

\subsection{How are iterated maps used to generate fractals?}
TODO

\subsection{What were Mandelbrot's contributions to fractal geometry?}
TODO

\subsection{How does bifurcation theory explain the transition from regular to chaotic behaviour?}
TODO

\section{Fractals in Nature}
\subsection{How do chaotic processes contribute to fractal patterns in nature?}
TODO

\subsection{How do coastlines display fractal patterns?}
TODO

\subsection{How do trees display fractal patterns?}
TODO

\subsection{How do mountains display fractal patterns?}
TODO

\section{Conclusion}

The dynamics of a system at each moment of time can be in one of these two states:
\begin{itemize}
    \item Chaos (unstable)
    \item Order (stable)
\end{itemize}


At either of those states you also need a perspective to be able to maximize your effectiveness and live optimally. These perspectives are Zooming out and Zooming in. 

You zoom out when the system is in a state of chaos. What that means is you try to grasp the bigger picture and understand why things unfold in the long run.

You zoom in when the system is in a state of order. What that means is you bring yourself to the present moment and try to take it in as much as possible. This includes when the economy is stable, when routines are predictable, and when life feels steady.

Chaos theory teaches us that there is no certainty in life, only possibility and patterns, and that is enough.


\nocite{*} 
\bibliographystyle{plain} 
\bibliography{references}

\end{document}
